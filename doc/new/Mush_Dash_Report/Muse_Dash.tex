\documentclass[12pt, a4paper]{article}

\usepackage[fontset=fandol]{ctex} 

% --- 常用数学与图形包 ---
\usepackage{amsmath, amssymb}
\usepackage{graphicx}
\usepackage{geometry}
\geometry{left=2.5cm, right=2.5cm, top=2.5cm, bottom=2.5cm}

% --- 代码高亮设置 (使用 listings,轻量级) ---
\usepackage{listings}
\usepackage{xcolor}

\lstset{
    basicstyle=\ttfamily\small, % 字体大小
    backgroundcolor=\color{gray!5}, % 背景色
    frame=single,   % 边框
    keywordstyle=\color{blue}, % 关键字颜色
    commentstyle=\color{green!50!black}, % 注释颜色
    stringstyle=\color{red}, % 字符串颜色
    numbers=left, % 行号在左边
    numberstyle=\tiny\color{gray}, % 行号样式
    breaklines=true, % 自动换行
    captionpos=b % 标题在底部
}

% --- 链接设置 ---
\usepackage[hidelinks]{hyperref}

\begin{document}

\begin{titlepage}
    \centering % 让所有内容居中
    \vspace*{1cm} % 顶部留白
    
    % 学校/课程名称
    \Huge \textbf{复旦大学课程项目报告} \\[1.5cm]
    
    % width=10cm 控制图片宽度,可以自己调整
    \includegraphics[width=10cm]{figure/fudan-name.pdf} \\[2cm] 
    
    % 标题
    \Large \textbf{FPGA Mush Dash} \\[2cm]

    \begin{center}
    \Large
    \textbf{小组成员:} \\[0.5cm] % 标题单独放
    \begin{tabular}{l l}        % {l l} 表示有两列,都是左对齐
        陈奕宽 & 25213040003 \\
        陈之逸 & 25213040070 \\
        林炫达 & 24212020013 \\
        孟思思 & 24212020144 \\
        薛文宵 & 24212020213 
    \end{tabular}
    \end{center}
    
    \vspace{2cm}
    \today
    \vfill % 自动填充底部空白
\end{titlepage}

% 生成目录
\tableofcontents
\newpage

% --- 正文开始 ---
\section{简介}

\subsection{项目背景与实验要求}
在嵌入式系统设计课程实验提供的三个选题方向(CPU扩展、VGA游戏、Web控制)中,我们选择了题目二:基于 FPGA 的 VGA 游戏设计与改进。
基于题目要求,本项目在 Altera DE2-115 FPGA 开发板上复刻了游戏Muse Dash,不仅在硬件层面上实现了游戏的判定与显示模块,而且还构建了一套 Python 环境的上位机工具链,打通了从谱面解析、算法处理、自动生成 ROM 代码到板级验证的完整闭环,实现了高度自动化的软硬件协同开发流程。

\subsection{系统整体设计}
本项目复刻了知名节奏游戏 \textbf{Muse Dash(喵斯快跑)} 的核心玩法。
\begin{itemize}
    \item \textbf{游戏机制}:系统在 FPGA 端复刻了经典的双轨道(上层/下层)判定机制。玩家需根据背景音乐的节奏,通过按键击打对应轨道上飞驰而来的敌人。
    \item \textbf{软硬协同架构}:
    \begin{itemize}
        \item \textbf{软件端(上位机)}:负责复杂的离线数据处理,包括音频播放、谱面文件的解析、特征提取、可视化分析、难度评估以及 Verilog ROM 文件的自动化生成。
        \item \textbf{硬件端(FPGA)}:负责毫秒级的实时交互响应,包括 TextLCD 的动态滚屏显示、七段数码管的实时计分反馈以及音频同步信号的触发。
    \end{itemize}
\end{itemize}

\subsection{项目分工与协作模式}
本项目采用了现代软件工程中的模块化与接口隔离设计模式,团队成员通过 Git 进行协作,通过标准化的 JSON 协议与 API 接口实现模块解耦。具体分工如下:

\begin{description}
    \item[陈之逸] 
    \begin{itemize}
        \item \textbf{软件框架设计}:设计上位机交互逻辑,实现模式切换、Quartus 自动化调用及音频预览功能,完成系统框架搭建。
        \item \textbf{Verilog编程与验证}:负责顶层模块的编写和全系统的上板校验。
    \end{itemize}

    \item[陈奕宽] 
    \begin{itemize}
        \item \textbf{Verilog编程与验证}:负责核心硬件模块的 RTL 设计与仿真。
        \item \textbf{报告撰写与演示}:负责 LaTeX 实验报告的撰写与排版优化,制作答辩演示文稿并负责项目最终展示。
    \end{itemize}
    
    \item[林炫达] 
    \begin{itemize}
        \item \textbf{谱面可视化分析}:实现 \texttt{chart\_analysis} 模块,开发谱面内容的深度统计逻辑与可视化绘图引擎。
        \item \textbf{数据验证}:通过统计数据(如 Note 总数、密度分布)对比硬件运行结果,辅助定位软硬件数据传输中的解析错误。
    \end{itemize}
    
    \item[薛文宵] 
    \begin{itemize}
        \item \textbf{核心引擎}:负责 \texttt{chart\_engine} 模块,实现谱面校验、BPM 更新及随机生成算法。
    \end{itemize}
    
    \item[孟思思] 
    \begin{itemize}
        \item \textbf{音频同步机制}:设计基于“空格键触发”的启动播放协议与握手信号。
        \item \textbf{软硬协同方案探索}:
    \end{itemize}
\end{description}

\subsection{工具链}
\begin{itemize}
    \item \textbf{版本控制}:Github
    \item \textbf{Verilog 编程}:VSCode + Vivado
    \item \textbf{Python 编程}:VSCode
    \item \textbf{行为级仿真}:Vivado
    \item \textbf{硬件验证}:DE2-115 开发板 + Quartus 平台
\end{itemize}

\section{软件设计}

\subsection{输入谱面格式}
我们设计了一种标准化的 txt 谱面格式。

\begin{description}
    \item[参数] 包含 BPM(曲速)以及每个音符的三个属性:\textbf{时间 (time)}、\textbf{类型 (type)}、\textbf{轨道 (track)}。
    \item[音符定义] \hfill
    \begin{itemize}
        \item \textbf{Time}:音符被判定的时钟周期数。
        \item \textbf{Type}:TAP (短按), HOLD\_START (长按开始), HOLD\_MIDDLE (长按保持)。
        \item \textbf{Track}:上轨道 (1) 或 下轨道 (0)。
    \end{itemize}
\end{description}

\subsection{谱面可视化分析模块 (Chart Analysis)}
\label{sec:chart_analysis}

本模块是 MuseDash 项目的重要组成部分,旨在对游戏谱面进行全面的统计分析与可视化展示。该模块能够自动扫描项目中的谱面目录,解析谱面文件,提取关键信息,并生成多种类型的统计图表和数据摘要,为谱面设计、难度评估和数据分析提供有力支持。

\subsubsection{系统架构设计}

\paragraph{模块结构}
谱面分析模块采用模块化设计,主要包含以下组件:
\begin{itemize}
    \item \textbf{主程序} (\texttt{chart\_analysis.py}):负责整体流程控制。
    \item \textbf{依赖管理} (\texttt{requirements.txt}):确保运行环境的一致性。
    \item \textbf{输出目录} (\texttt{outputs/}):存储所有生成的分析结果,包括协议文件、统计摘要和可视化图表。
\end{itemize}

\paragraph{核心类设计}
系统采用面向对象的设计思想,将功能划分为三个核心类:
\begin{enumerate}
    \item \textbf{ChartParser(谱面解析器)}:负责读取和解析谱面 TXT 文件,从文件第一行提取 BPM 信息,逐行解析音符事件,并计算谱面总时长。
    \item \textbf{ChartAnalyzer(谱面分析器)}:对解析后的数据进行多维度统计分析,包括总音符数统计、类型分布统计、密度曲线计算、轨道分布统计、时间分布数据提取以及难度曲线计算。
    \item \textbf{ChartVisualizer(谱面可视化器)}:负责将统计数据转化为直观的可视化图表,生成六种不同类型的图表文件。
\end{enumerate}

\subsubsection{核心实现流程}

\paragraph{1. 谱面解析流程}
系统首先扫描 \texttt{charts/} 目录下的所有谱面子目录,对每个谱面目录进行以下处理:检查谱面 TXT 文件是否存在,调用 \texttt{chart\_check()} 函数进行格式校验,使用 \texttt{ChartParser} 解析谱面文件。
解析过程中,系统读取文件第一行提取 BPM 值,使用正则表达式逐行匹配音符事件格式 \texttt{(时间,类型,轨道)},将所有音符记录到列表中,并计算最大时间作为谱面时长。解析完成后,系统将 BPM、音符列表和时长信息传递给分析器。

\paragraph{2. 统计分析流程}
\texttt{ChartAnalyzer} 接收解析后的数据,执行全面的统计分析:
\begin{itemize}
    \item \textbf{音符计数}:在总音符数统计方面,系统只统计 \texttt{tap} 和 \texttt{hold\_start} 类型的音符,避免 \texttt{hold\_mid} 音符的重复计算,确保统计结果的准确性。
    \item \textbf{密度曲线计算}:采用时间窗口方法,将谱面时长划分为多个时间窗口,窗口大小根据谱面总时长动态调整(公式为 $\max(100, \text{duration}/100)$)。随后统计每个窗口内的音符数量,生成时间-密度映射关系,有效反映谱面在不同时间段的音符密度变化。
    \item \textbf{难度曲线计算}:这是系统的核心功能之一。为了量化谱面的瞬时难度,系统建立了一个综合考量音符类型、轨道并发数及密度的多因子加权模型。难度分数 $Score(t)$ 的计算公式如下:
    \begin{equation}
        Score(t) = \left(\sum w_{type}\right) \times \left[1 + 0.2 \times (N_{track}-1)\right] \times \left[1 + 0.1 \times (N_{notes}-1)\right]
    \end{equation}
    其中各项参数定义为:
    \begin{itemize}
        \item $\sum w_{type}$ \textbf{基础权重和}:窗口内所有音符权重的总和。具体权重设定为:Tap=1.0,Hold Start=1.5,Hold Mid=0.3,Hold End=0.5。
        \item $N_{track}$ \textbf{轨道复杂度}:当前时间窗口内活跃的轨道数量。当多轨并发时,难度系数按线性增长(每增加一轨权重增加 0.2)。
        \item $N_{notes}$ \textbf{密度因子}:当前窗口内的音符总数。音符越密集,手速要求越高,难度系数随之增长(每增加一个音符权重增加 0.1)。
    \end{itemize}
    最终生成的难度曲线不仅反映了谱面的张力变化,系统还会据此自动标记全曲的“最难点”(Peak Difficulty)。
\end{itemize}

\paragraph{3. 可视化生成流程}
\texttt{ChartVisualizer} 根据分析结果生成六种不同类型的可视化图表,所有图表均采用 200 DPI 的高分辨率输出:
\begin{itemize}
    \item \textbf{音符类型数量饼图}:展示各类型音符的绝对数量(实际数字)。
    \item \textbf{音符类型占比饼图}:展示各类型音符的相对比例(百分比)。
    \item \textbf{密度曲线图}:使用折线图和填充区域展示谱面随时间变化的音符密度。
    \item \textbf{轨道分布柱状图}:展示每个轨道的音符数量,并在柱状图上直接显示数值标签。
    \item \textbf{音符时间分布直方图}:展示音符在时间轴上的分布情况,根据谱面时长动态调整 bins 数量。
    \item \textbf{难度曲线分析图}:综合考虑密度、音符类型复杂度和轨道分布,显示平均难度线和峰值标记。
\end{itemize}
图表设计采用了现代配色方案(红、青、蓝渐变色),并优化了字体大小和粗细,添加了阴影效果增强立体感,曲线图和柱状图添加网格线以辅助数值读取。

\paragraph{4. 数据输出流程}
系统生成两种主要的数据输出文件:
\begin{itemize}
    \item \textbf{summary.json}:包含谱面的核心统计信息,包括曲目名、BPM 值、谱面时长、总音符数、Tap/Hold\_start 音符数量、各类型音符数量分布、轨道分布、密度峰值和平均密度。
    \item \textbf{protocol.json}:作为协议文件,汇总所有谱面的信息,包括每个谱面的名称、生成的图表文件列表、summary 文件路径,以及可选的 BPM、时长、文件夹名和音频文件名。
\end{itemize}

\subsubsection{技术实现细节}

\paragraph{依赖库选择}
\begin{itemize}
    \item \textbf{matplotlib}:作为主要的图表生成库,支持多种图表类型,输出质量高。
    \item \textbf{numpy}:用于数值计算和颜色生成,提供高效的数组操作和数学运算支持。
    \item \textbf{json/pathlib/re}:分别用于 JSON 读写、跨平台路径处理及正则表达式解析。
\end{itemize}

\paragraph{谱面文件格式解析}
谱面文件采用文本格式,第一行为 BPM 信息(\texttt{bpm=数值}),后续每行表示一个音符事件 \texttt{(时间,类型,轨道)}。其中时间为整数,类型包括 \texttt{tap}、\texttt{hold\_start} 等,轨道为 0 或 1。系统使用正则表达式精确匹配这种格式,确保解析的准确性。

\paragraph{图表尺寸适配}
为了适配前端显示需求,系统对不同类型的图表采用不同的尺寸策略:
\begin{itemize}
    \item \textbf{饼图}:使用 $6 \times 6$ 英寸的正方形尺寸,不使用 \texttt{bbox\_inches='tight'} 以保持固定比例。
    \item \textbf{其他图表}(曲线、柱状、直方):统一使用 $9 \times 6$ 英寸的 3:2 比例,与前端容器的 $180 \times 120$ 像素比例匹配,确保图表在前端完整显示而不被裁剪。
\end{itemize}

\paragraph{特殊处理机制}
系统对 Random 谱面采用与其他谱面相同的处理逻辑,保证一致性。在统计总音符数时,特意剔除 \texttt{hold\_mid} 音符以避免重复计数。同时,系统明确区分了 note\_count(整数)和 note\_density(百分比)两种饼图,满足不同分析需求。对于空数据情况,系统会生成空图表占位,避免程序异常。

\subsubsection{使用方式与前端集成}

\paragraph{环境配置与运行}
使用前需执行 \texttt{pip install -r requirements.txt} 安装依赖(要求 Python 3.6+)。在目录下执行 \texttt{python chart\_analysis.py} 即可自动扫描 \texttt{charts} 目录,处理完成后结果保存在 \texttt{outputs/} 中。

\paragraph{输出文件与前端集成}
输出目录包含 \texttt{protocol.json} 协议文件、每个谱面的 \texttt{summary.json} 及六种 PNG 图表。前端系统通过读取 \texttt{protocol.json} 获取所有谱面信息列表,进而加载对应的图表(标准 PNG 格式,直接嵌入 HTML)和统计摘要(JSON 格式,便于 JavaScript 解析)。

\subsubsection{功能总结与扩展方向}
当前版本已实现完整的谱面分析功能,包括自动扫描解析、多维度统计(总数、类型、密度、轨道、时间、难度)、高质量可视化生成及协议文件输出。系统设计合理,功能完善。

未来扩展方向包括:
\begin{itemize}
    \item 支持多 BPM 谱面并绘制 BPM 变化曲线。
    \item 使用 Plotly 生成可交互的 HTML 图表。
    \item 添加连击数、最大连击、平均间隔等更多统计指标。
    \item 生成热力图,展示音符在时间和轨道上的热力分布。
\end{itemize}

\subsection{谱面转 ROM}

\subsection{随机谱面生成与校验}

\subsection{音乐播放}

\subsection{软硬协同方案探索}


\section{硬件设计}

\paragraph{显示与输入}
\begin{itemize}
    \item \textbf{音符类型}:`O` 代表 TAP(点击),`<` 代表 HOLD\_START(长按开始),`=` 代表 HOLD\_MIDDLE(长按保持)。
    \item \textbf{按键映射}:左起第一个按钮控制上轨道,第二个控制下轨道;右起第一个按钮为重启 (restart),右起第一个开关为复位 (\texttt{rst\_n})。
    \item \textbf{反馈}:七段数码管显示判定结果(PERFECT, GOOD, MISS/FL)和实时累计分数。
\end{itemize}

\subsection{顶层模块}
顶层模块负责连接各个子模块并执行基本逻辑运算(如复位信号整合)。它协调各模块(如时钟分频、地址生成、队列、判定、显示等)协同工作,简化了整体架构。

\subsection{分布式架构 (Distributed Architecture)}
为了便于调试和分工,我们将设计划分为 14 个子模块,包括:

\begin{description}
    \item[Clk\_Div] 生成可变频率时钟。
    \item[LFSR] 随机数生成(用于测试)。
    \item[Debouncer] 按键消抖。
    \item[Queue] 数据队列处理。
    \item[Judgement] 核心判定逻辑。
    \item[TextLCD \& 7Seg] 显示模块。
    \item[Score \& Accumulator] 分数转换与累加。
\end{description}

\subsection{层次化设计}
主要体现在加法器和累加器的设计上:
\begin{itemize}
    \item \textbf{结构}:1位 BCD 加法器 $\rightarrow$ 4位 BCD 加法器 $\rightarrow$ 16位累加器。
    \item \textbf{预加法 (Pre-Addition)}:在 \texttt{ScoreConversion} 模块中先将上下轨道的分数相加,再传入累加器,减少了累加器的端口复杂度。
\end{itemize}

\subsection{可调节时钟分频器}
\texttt{clk\_div} 模块通过参数 \texttt{div\_cnt} 将 50MHz 输入时钟分频为低频信号(如 10Hz/100ms)。该参数在顶层模块中根据 BPM 动态计算,使得 C++ 程序可以轻松改变游戏速度。

\subsection{消抖}
利用计数器和状态标志来过滤机械按键的抖动。只有当输入信号稳定一段时间(计数器归零)后,输出信号才会更新。这确保了输入信号的准确性。

\subsection{随机数生成}
使用 13位 LFSR(线性反馈移位寄存器)生成伪随机序列。主要用于在代码未完成时的随机输入测试。

\subsection{TextLCD 显示}

\subsubsection{队列设计}
接收 ROM 数据并在 \texttt{clk\_div} 驱动下进行移位,模拟音符滚动效果。最左侧数据用于判定和显示。

\subsubsection{Text LCD 设计}
基于 HD44780 控制器。状态机循环执行:空闲 $\rightarrow$ 清屏 $\rightarrow$ 设置模式 $\rightarrow$ 设置 DDRAM 地址 $\rightarrow$ 写入数据。将音符映射为 ASCII 码写入 DDRAM。

\subsubsection{双频率设计}
引入了 \texttt{clk\_buf}(屏幕帧率)独立于 \texttt{clk\_div}(游戏逻辑/BPM)。这解决了帧率过低导致显示闪烁或上下轨道不同步的问题。

\subsection{七段数码管显示}
由于 DE2-115 板载数码管是独立控制的,不需要多路复用。我们设计了纯组合逻辑电路来显示数字(0-9)和字符(PF, GO, FL)。

\subsection{判定}
这是最核心的模块,负责接收音符、检测按键并输出结果。

\subsubsection{硬件判定标准}
\begin{itemize}
    \item \textbf{TAP/HOLD\_START}:在判定窗口内,若按键上升沿落在窗口中间 $\pm 1/2$ 区域为 \textbf{PERFECT},否则为 \textbf{GOOD}。超出窗口为 \textbf{MISS}。
    \item \textbf{HOLD\_MIDDLE}:必须在前半个判定周期内保持按住。如果松开过早则为 MISS。
\end{itemize}

\subsubsection{信号判定}
引入 \texttt{judged} 信号标志当前音符是否已完成判定,防止重复计分。
\textbf{逻辑}:检测到有效按键或超时未按,信号置 1;判定周期结束或无音符时,信号置 0。

\subsubsection{FSM 设计}
状态机包含 \textbf{NO\_NOTE, GOOD, PERFECT, MISS} 四个状态。
\textbf{特殊机制}:为了解决组合逻辑传播延迟问题,我们强制 FSM 在音符判定中间的 \texttt{clk\_div} 上升沿短暂返回 NO\_NOTE 状态。

\subsubsection{Vivado 仿真}
在上板前进行了详细的行为级仿真,验证了不同音符和按键时机的判定逻辑。

\subsection{分数累加}
\begin{itemize}
    \item \textbf{BCD 加法器}:采用级联的 1位 BCD 加法器构建 4位加法器。当和大于 9 或有进位时,进行 +6 修正。
    \item \textbf{预加法}:在分数转换模块先计算 \texttt{score\_up + score\_down},然后再输入累加器,避免了设计复杂的三输入加法器。
\end{itemize}

\section{讨论}

\subsection{设计技巧}
\begin{itemize}
    \item 避免生成锁存器(Latches):在组合逻辑中完整覆盖 \texttt{if-else} 或 \texttt{case} 分支。
    \item 优先使用 \texttt{assign} 语句代替 \texttt{always} 块中的组合逻辑。
    \item 规范命名(如 \texttt{rst\_n} 表示低电平复位)和注释。
    \item 时序逻辑中使用非阻塞赋值 (\texttt{<=})。
\end{itemize}

\subsection{判定设计演变}
最初尝试分离“监控 FSM”和“结果 FSM”,因过于复杂而废弃。最终简化为单 FSM 设计。仿真无法覆盖所有错误,部分问题是在上板测试中发现并打补丁修复的。

\subsection{边沿捕捉}
在硬件中捕捉跨时钟域信号的上升沿非常困难。我们通过经验法则(试错法)决定是将信号延迟 1 个周期还是 2 个周期来生成脉冲信号。

\subsection{信号穿透}
这是一个关键问题。\texttt{cur\_note} 信号在切换时会出现毛刺(Glitch),导致 HOLD\_MIDDLE 信号在时钟偏差下“穿透”到前一个周期,造成错误判定。
\textbf{解决方案}:在判定区间的中间点,强制将状态置为 \textbf{NO\_NOTE} 一次。这相当于一次软复位,阻断了错误信号的传播。

\subsection{软硬件不一致性}
Vivado 仿真完美,但上板出现问题(如 FSM 卡死在 NO\_NOTE,或按键对 TAP 无响应)。由于无法定位根源,只能通过添加补丁代码(如强制状态跳转)来解决。

\subsection{Quartus 调试}
使用了 Signal Tap Logic Analyzer 进行片上调试。配置采样深度(2K)和触发条件(Basic AND/OR),观察实时波形以定位问题。

\section{总结}
我们成功在 DE2-115 上实现了一个基于 Muse Dash 的节奏游戏。

\section{未来展望}
\begin{enumerate}
    \item \textbf{消抖延迟优化}:目前的消抖模块引入了 $>20$ms 的延迟,对于判定区间仅 $\pm 50$ms 的节奏游戏影响较大。未来需要在判定逻辑中补偿这一延迟。
\end{enumerate}

\end{document}
